\documentclass{tahquitz}

\begin{document}

\tableofcontents

\vfill\pagebreak

%=====================================================================

\climb[1]{Fingertip Traverse}{5.4} %%%%%%%%% [1] is because this is the first one in the book

\topo{0.45}{fingertip_traverse}

An easy Tahquitz classic, this was the original climb used to define
5.1 on the Yosemite Decimal Scale. By modern standards most of the
climbing is 4th or easy 5th class, but the namesake traverse is more
like 5.4. Hike up to Lunch Rock and then head to the right along the
trail. The first pitch starts directly from the trail, near a 
truncated tree, with the first belay easily visible. Pro to 2". FA
1936, Jim Smith, Bob Brinton, Arthur Johnson, and William Rice.

\pitch{1} Scramble up to a large double oak tree in an alcove, the infamous
``ant tree.'' Anchor at the lower part of the tree to avoid stirring up
the ants. 30 m, 4th class.

\pitch{2} Move up the left side of the alcove's left wall, where it is
shortest, stemming against the opposite wall and then against one of
the tree branches. Protect with a 0.3" cam (blue C4) in a small
finger crack before moving onto the rock. (This can also be done by
climbing higher up the tree, or by climbing the broken overhang at
the back of the alcove. These options can be protected with a 2" cam
in the crack a the left corner.) Continue up easy terrain. 30 m, 5.0
except for the committing move out of the tree.

\pitch{3} Climb a straightforward layback (5.3) with good protection and
stances. Belay at a tiny ledge on a slab (small gear). 30 m

\pitch{4} Go around a corner and traverse along a finger-width crack,
smearing on the sloping face below, 5.4. Protect using the trees and
gear 1-2" and smaller. Continue up 5.0 climbing to Lunch Ledge (gear
belay). Here the route joins up with Angel's Fright. 40 m

\pitch{5} Climb up from the middle of the ledge. (Linking P5 and P6 will
give unacceptable rope drag unless you plan carefully.)

\pitch{6} Walk along a narrow flake that is easily protected. At a small
tree, head up a slab with an orange bolt. 5.4

\variation{1} Head right into a 5.1 gully, then turn left at an overhang.

\variation{2} Climb a very thin finger crack, punctuated with pin scars. 5.6

Walk uphill to the standard friction descent.

%=================================================================================

\climb{White Maiden's Walkway}{5.4}

\topo{0.35}{maiden}


A Tahquitz classic. Difficult routefinding. The route ends not far
below the summit of Tahquitz and is one of the longer routes on the
rock. Start from the left side of the Maiden Buttress, which can be
approached either from the North Side Trail or the Lunch Rock Trail.
Pro to 3", or 4" for var 2 of P2.

\pitch{1} The obvious way up is a gully with pine trees in it. Scramble up
class 4 rock to a belay at a dihedral. The first move is difficult;
step left first using a hidden fingertip hold for balance, then mantle
up into the crack.
After the first pine tree, enter a gutter heading up
and to the left, but then immediately exit the gutter to the right to
avoid getting the rope snagged in brush. Continue past a big pine
tree and belay near a pair of small trees. 5.0 except for the first
move

\pitch{2} Traverse to the right, over a saddle behind a boulder, and make
an exposed step around with friction into a gutter.  Protect with a
\#3 TCU. (It may be possible to bypass the friction move by downclimbing
slightly, but I haven't tried it.) Head up the 5.0 gutter and watch for an exit to a large saddle
on the right. A common mistake is to overhsoot the saddle instead of
exiting.

\variation{1} A clean hand and foot crack just left of the small deciduous
tree. 5.7

\variation{2} A ragged fist- and off-width crack heading straight up from
the small deciduous tree. 5.6

\pitch{3} From the center of the saddle, head straight up toward a small
tree. Near the second fixed pin, you run out of easy ways up, the
only apparent option being a very difficult mantling move up to a
small sloper hold. Grope for a series of excellent holds that allow
you to easily move up and to the right. Come back to the tree. Place
gear to route the rope so as to avoid damaging the tree by dragging
the rope over it. Continue left across the ledge to a belay at
another tree. 5.4, 35 meters

\pitch{4} Climb 15 meters up a flake. As the flake levels out and
disappears, look for a large, loose conch-shaped block close enough
on the left to reach out and touch. Step across two loose blocks and
make a committing step around a corner to a stemming stance, then
reach for a tree. Head up a 5.0 gully. Optional belay at any of
several spots with trees or gear anchor possibilities -- or continue
to a large tree.

\pitch{5} Traverse right toward another tree, but head upward before it.
Gear belay in a one-inch vertical crack before the big gully. 4th
class.

\pitch{6} For the standard 5.3 finish, head up and out of the gully onto a
ledge with a claustrophobic overhang. At a large block, detour to
escape the overhang. Move all the way out to the rightmost side of
the block. Get your right foot up on a small step-stool,
side-pull on a hold, and mantle up onto the block.

Either go down to the friction descent or go up and over the summit
and come down the north gully.

%=================================================================================

\climb{Northeast Face, East Variation}{5.6}

\topo{0.25}{northeast_face_east}

This climb is in the middle of the northeast face, to the left of the
Larks and to the right of El Grandote. A prominent inverted ``Y'' is
formed by right-facing dihedrals, and this variation heads up its
left fork. Approach via the North Side Trail. The ``Y'' is difficult to
see through the trees from the trail, but the north face lies below
the prominent gendarme at the top of the Larks, and this climb is on
the left side of that face. A fun route that is easily protected and
can be climbed on a Saturday morning without being in a conga line. A
possible negative is that there is a certain sameness to a lot of the
climbing on pitches 2-4. Sustained 5.5-5.6 climbing on pitches 2
through 5. Pro to 3", with  a 4" cam repeatedly coming in handy.

\pitch{1} Climb flakes to the left of the dihedral, enter the left-facing
part of the dihedral briefly, and belay at a small ledge. Ignore the
bolts to the right, which are for the slab climb Grace Slick. 60 m,
5.3.

\pitch{2} Continue up the crack. Insert into the main right-facing part of
the dihedral and climb a wide (knee jam) crack. End at a bolted
anchor on a huge ledge. 5.5

\pitch{3} Continue up the dihedral, toeing in to a thin crack at the
corner, to a series of overlaps at the crotch of the ``Y.'' Traverse
past a small tree, then move up through 5.6 broken terrain to a
belay. The crux of the climb is at the end of this pitch and the
beginning of the next one. Vogel and Gaines describe a variation that
swings around farther to the right, but I've heard it's harder.
Higher up is an alternative belay at a tree around a corner to the
right, but this is a poor choice due to rope drag.

\pitch{4} Head up the stem of the ``Y'' toward overhangs. Stay in the
dihedral, and don't be lured off route by the fixed piton above and
to the left. Belay as close as possible to the overhangs in order to
complete the following pitch with a 60 meter rope. A fixed pin can be
backed up with 3" or 4" cams.

\pitch{5} Pull through gaps in a set of steep blocks and irregular
overhangs. The  moves are strenuous but easily protected, with big
holds. After this the climbing gets much easier for the rest of the
route. Head for a shady belay at a pine tree on a big ledge. 60 m

\pitch{6} Continue to the right up 4th class terrain to the summit ridge,
or traverse a delicate 5.3 ledge to the left for more direct
insertion into the north gully.

%=================================================================================

\climb{The Trough}{5.3}

\topo{0.3}{trough}

This climb was the original definition of 5.0 on the Yosemite Decimal
Scale, but is now considered about 5.3. 3-5 pitches. Gear to 3",
slings for trees. Singles of cams are enough.

Approach: From the bottom end of the parking area at Humber Park,
take the Ernie Maxwell trail and then turn left at the signpost for
the climber's trail. Climb to Lunch Rock. Continue past Lunch Rock to
the foot of Tahquitz Rock. Turn left, and crawl up through a tunnel
behind a tree. Follow the ledge until it ends.

\pitch{1} Follow a crack up and around a corner to the left. Fingers and
friction initially, then off-width. Pass through a large, sloping,
triangular ledge and insert into the obvious trough. Belay at a small
ledge on the right. 45 m

\pitch{2} Continue up the trough. Near the top is more difficult climbing
including smearing and squeezing through a narrow gap. At the top,
the trough steepens and ends. Move up and over to the right here,
onto Pine Tree Ledge. 45 m.

\pitch{4} Climb a face a couple of feet to the right of the pine tree,
aiming for the small oak tree that is visible against the sky. It's
difficult to find gear placements here that aren't cracks behind
flakes, but it is possible to sling a series of tiny trees. The slope
moderates and turns into a gully. Beyond the oak tree is a short
section of easy climbing which ends at a huge pine tree. 55 meters,
5.0 and 4th.

- From the pine tree, climb up class 2 slab to the left, staying on
the crest of the ridge to avoid exposure and steeper slopes. About 60
meters from the pine tree, you reach an area of boulders and bushes
(before the summit).

- The standard friction descent route (class 3) is on the south side
of Tahquitz Rock. Descend a chimney behind a house-sized boulder.
Zigzag down a series of ledges, then head for a double tree. Traverse
slabs to get to the trail.

- Return on the climber's trail, staying close to the foot of the
rock except to avoid the slabs near the bottom of the Ski Tracks area.

%=================================================================================

\climb{West Lark}{5.3}

\topo{0.2}{west_lark}

A low-angle ascent up an obvious crack, with straightforward routefinding.
A fall might mean hitting a ledge. The few harder moves
are protectable. Comfy belays are scarce, so when you see one, use
it. Pro to 2". Singles of cams should be enough for most leaders.

The start lies at  the extreme right side of the north face, just to
the left of the north buttress. Approach via the North Side Trail,
leaving the trail to approach the base of the climb by threading your
way between talus piles. The gendarme near the top is the most
visible landmark through the treetops.

To find the beginning of the climb, get up close and look for the
obvious easy start in a recess, with the bush and short friction
traverse shown in the topo. Nearby to the right is difficult or
impossible climbing consisting of a left-facing dihedral ending in an
overhang. 

\pitch{1} Climb past an inconvenient bush and left a few feet across slab.
(A direct start, to the left, is probably also possible, but I haven't tried it.)
Insert into a crack and head up. $\sim 35$ m

\pitch{2} Continue up the crack to a ledge on the right.  $\sim 30$ m

\pitch{3} Continue up through a gap in an overhang and belay above it.
It is possible to belay below the overhang instead, but then it becomes
difficult to do the next two pitches with a 60-meter rope. $\sim 30$ m

\pitch{4} Pass through a gap in the overhang and continue to a hanging
belay next to a right-facing dihedral. The belay needs to be right near the
overhang if P5 is to be possible with a 60-meter rope. 50 m

\pitch{5} Up to a ledge below a huge gendarme. 60 m

\pitch{6} Continue straight up, on the right side of the gendarme.
Pass between the gendarme and its little sister, a rock shaped like
the prow of a ship. Face and chimney climbing to the top.

\variation{1} It is also possible to go to the left of the gendarme.
Rope drag may be a problem, and the crux move may be difficult to protect
without depending on bad rock.

Gain the summit ridge and head down left into the north gully descent route,
or insert into the gully more directly by scrambling.

\myfooter\vfill %%%%%%%%%%%%%% because this is the last topo

\end{document}
